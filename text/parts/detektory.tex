\chapter{Pasivní detektory používané k monitorování kosmického záření}\label{sec:detektory_detektory}

%vyhodnocování OSLD probíhá pomocí referenčního zdroje (ten vztah asi nepsat) (dosis, s. 7)

%napsat z ceho jsou vyrobene (nebo spis co presne jsou) TLD pouzivane NPI (prasek, krystal atd.)

%HSP1000 -> jediny v Evrope!!!!!!!!!!! (viz stranky ODZ UJF AVCR)


%Nevýhodou pasivních detektorů je skutečnost, že po každém měření musely být dopraveny zpět na Zem do příslušné laboratoře a teprve tam byly vyhodnoceny; avšak oproti aktivním detektorům nemusejí být napájeny proudem, což je v prostředí ISS velmi praktické. %toto asi až to další kapitoly o detektorech

