\chapter*{Závěr}
\markright{Závěr}
\addcontentsline{toc}{chapter}{Závěr}
V této práci byla probrána problematika kosmického záření v blízkém okolí Země a jeho monitorování. Bylo pojednáno o pasivních detektorech, které se využívají k jeho detekci. Stručně jsem zmínil jejich způsob fungování a podrobně jsem rozepsal metodiku vyhodnocování detektorů používaných Oddělením dozimetrie záření Ústavu jaderné fyziky AVČR. Ve stěžejní části byly popsány experimenty DOSIS a DOSIS3D, které se zabývají stanovením prostorové distribuce dávky v ISS modulu Columbus. Jejich výsledky byly konfrontovány s daty jiných, podobných experimentů. Nakonec jsem vyhodnotil tři detektory stop TASTRAK, které byly umístěné od prosince 2015 do června 2016 v rámci experimentu DOSIS3D v modulu Columbus. K tomuto účelu jsem napsal skript v programovacím jazyce Python.
Vyhodnocování detektorů jsem důsledně popsal. Výsledkem je vypočítaná celková absorbovaná dávka a dávkový ekvivalent (resp. příslušné příkony) z každého detektoru. Pro každý detektor
jsem dále vytvořil čtyři $\LET$ spektra -- pro počet detekovaných částic, pro diferenciální fluenci, pro dávku a pro dávkový ekvivalent. Dávkové příkony prvního, druhého a třetího detektoru jsou $D_1=(30\pm6)\ \mu$Gy/den, $D_2=(28\pm6)\ \mu$Gy/den a $D_3=(25\pm5)\ \mu$Gy/den. Příkony dávk. ekvivalentů z odpovídajících detektorů jsou $H_1=(500\pm100)\ \mu$Sv/den, $H_2=(500\pm100)\ \mu$Sv/den, $H_3=(400\pm100)\ \mu$Sv/den. Dávkové příkony z termoluminiscenčních detektorů vycházejí zhruba o jeden řád vyšší. Tento rozdíl je dán necitlivostí vyhodnocených detektorů stop vůči částicím s $\LET<10$ keV/$\mu$m, nižší účinností TLD vůči částicím s vyšším $\LET$ a také tím, že vyhodnocené detektory stop byly leptány pouze jednou, tj. byly zobrazeny pouze některé stopy. Kosmické záření je
hlavně tvořeno protony (malé $\LET$), nicméně částice s velkým $\LET$ přispívají k $H$ nezanedbatelně (podobně nebo více jak nízké částice). $\LET$ spektra jsou k vidění v předchozí kapitole.

V budoucnu lze detektory stop leptat podruhé, čímž se zobrazí další stopy. Vypočte se opět dávka, dávk. ekvivalent, a poté se přičtou k výsledkům prvního leptání. Kompletní výsledky z detektorů stop se dále zkombinují s výsledky z TLD, a tak se určí celková absorbovaná dávka a dávk. ekvivalent. 

Dosavadní výsledky experimentů DOSIS a DOSIS3D poukazují na nutnost dlouhodobého souvislého měření kosmického záření v blízkém okolí Země, mají-li být prozkoumány všechny vlivy na jeho tok daným místem a na velikost obdržených dávek v daném místě.

Cílem této práce bylo udělat rešerši na problematiku kosmického záření v blízkém okolí Země a jeho monitorování pasivními detektory, seznámit se podrobně s probíhajícími experimenty zkoumající kosmické záření v blízkosti Země (hlavně s DOSIS a DOSIS3D), obeznámit se s metodikou a vyhodnocováním detektorů stop, analyzovat několik detektorů. Všeho bylo dosaženo. 
