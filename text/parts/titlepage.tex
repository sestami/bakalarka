\begin{center}
{\LARGE ČESKÉ VYSOKÉ UČENÍ TECHNICKÉ V PRAZE\\}
	\vspace{10pt}
{\large FAKULTA JADERNÁ A FYZIKÁLNĚ INŽENÝRSKÁ\\}
{\large KATEDRA DOZIMETRIE A APLIKACE IONIZUJÍCÍHO ZÁŘENÍ\\}
	\vspace{40pt}
	\includegraphics[scale=0.7]{cvut.png}

	\vspace{40pt}
{\Huge \textbf{BAKALÁŘSKÁ PRÁCE\\}}
	\vspace{10pt}
{\LARGE \textbf{Prostorová distribuce dávky uvnitř Mezinárodní kosmické stanice\\}}
	\vspace{150pt}

\end{center}
{\large
\begin{tabular}{p{4cm} p{8cm}}
Autor: & Michal Šesták\\
Vedoucí práce: & Ing. Iva Ambrožová, Ph.D.\\
Praha, 2017 & \\
\end{tabular}
}
\newpage
\vspace*{\fill}
\section*{Prohlášení}
Prohlašuji, že jsem svou bakalářskou práci vypracoval samostatně a použil jsem pouze podklady uvedené v přiloženém seznamu.\\[10pt]
V Praze dne \\[10pt]
\newpage
\vspace*{\fill}
\section*{Poděkování}
Děkuji Ing. Ivě Ambrožové, Ph.D. za vedení mé bakalářské práce, za cenné rady a připomínky, které tuto práci obohatily.
\newpage
\begin{tabularx}{\textwidth}{>{\itshape}l X}
  Název práce: & \textbf{Prostorová distribuce dávky uvnitř Mezinárodní kosmické stanice}\\
  Autor: & Michal Šesták\\
  Obor: & Dozimetrie a aplikace ionizujícího záření\\
  Druh práce: & Bakalářská práce\\
  Vedoucí práce: & Ing. Iva Ambrožová, Ph.D.\\ 
               & Oddělení dozimetrie záření, Ústav jaderné fyziky AV ČR, v.v.i., Akademie věd České republiky\\
  Konzultant: & <++>\\
  Abstrakt: &  	
Cílem práce bude studovat prostorovou distribuci dávkových veličin na různých místech ISS pomocí kombinace termoluminiscenčních detektorů a detektorů stop v pevné fázi, které byly poustupně umístěny na ISS v posledních několika letech. Student se seznámí s metodikou vyhodnocování detektorů, včetně měření a zpracování dat. \\
  Klíčová slova: & <++>
\end{tabularx}
\newpage
\begin{tabularx}{\textwidth}{>{\itshape}l X}
  Title: & \textbf{Dose distribution inside the International Space Station}\\
  Author: & Michal Šesták\\
  Abstract: & <++>\\
  Key words: & <++>
\end{tabularx}
\newpage
